Short read genome mapping is an important application in genomics. It is an algorithm to locate a short read of DNA on the full genome. If enough of such short reads are mapped, some interesting results can be achieved. For example, if enough reads are mapped, we can extrapolate the full genome of the organism and all DNA mutations can be detected. Moreover, the amount of reads in one place (referred to as the reading depth) can give useful information, e.g. the presence of extra DNA like a trisomy of chromosome-21 in Down syndrome patients.

Typically, each read is compared with the whole genome in a local alignment, for example with the Smith-Waterman algorithm. As an output, we get the position in the human genome and alignment with its score (how well the sequence fits in that spot). This practice is commonly referred to as \emph{Mapping to reference genome}.

A full genome cannot be determined immediately, because the machines that determine this sequence can only handle short reads consisting of a few hundred bases. DNA sequencing machines are currently capable of producing millions of reads per day, and their throughput is growing at an exponential rate. This exponential growth should be accompanied by an improvement in genome mapping techniques, to keep up with the throughput of these machines. However, most of the current software tools used for genome mapping are run on traditional CPUs. 

If we analyze the S-W algorithm, we can see that the value of each cell in the matrix is only dependent on the left-upmost 3 cells. Therefore, it leads us to believe that this algorithm can be accelerated on other hardware solutions such as an FPGA since the S-W algorithm is heavily parallelizable.

In most clinical applications where mapping to a human reference genome is used, the number of reads to be compared with the genome is in the millions, which further increases the demand to speedup the process of mapping the reads in the genome, to decrease the time-to-result.

In literature accelerating short read alignment on FPGAs was described to be 28 times faster in respect to CPUs and 9 times to GPU \cite{FPGAacc}. showed a speedup of 5.6x to 71.3x dependent on the accuracy of the alignment\cite{FPGAacc2}).

The idea of this thesis was to implement the Smith-Waterman alignment algorithm on an MPSoC, which contains both programmable FPGA hardware and an ARM processor in 1 chip. As a target board, the ZCU 104 evaluation kit was chosen.

As a starting point, a software implementation was implemented on the ARM processor. After running the software implementation with the unmapped sequences from the SARS-CoV-2 (coronavirus) as a sample set, we obtained a dataset of mapped reads. The reads were also mapped by using the Galaxy (Bowtie 2) online tool. We observed that our implementation is working correctly since the reading depth graphs are approximately the same. Furthermore, we were able to detect and identify several DNA mutations in the coronavirus DNA sequenced by the University of Washington in respect to the reference sequence from Wuhan (China).

Some reads showed different results in comparison with the Galaxy online tool, but this will probably be because the parameters used are a bit different. However, the important part is that the reading depths are the same, as well as the consistently mutated bases marked in the genome that were detected.

After implementing the Smith-waterman matrix fill-in in the FPGA hardware, we achieved speedup of 4.41 in comparison with an implementation running fully on the ARM processor.

\textbf{Keywords}: Smith-Waterman algorithm, short read genome mapping, accelerator, MPSoC, FPGA, HLS