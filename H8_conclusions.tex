%%%%%%%%%%%%%%%%%%%%%%%%%%%%%%%%%%%%%%%%%%%%%%%%%%%%%%%%%%%%%%%%%%% 
%                                                                 %
%                            CHAPTER                              %
%                                                                 %
%%%%%%%%%%%%%%%%%%%%%%%%%%%%%%%%%%%%%%%%%%%%%%%%%%%%%%%%%%%%%%%%%%% 

\chapter{Conclusion and future research}
\label{ch:Conclusions}

\section{Conclusion}

//TODO

\section{Future work}

//TODO

Gap opening and extension

In a sequence, a base may be suddenly marked with an 'N' character, which means after the primary processing the process was unable to identify the base. Because of the way the sequencing machines work, this mostly happens at the end of the sequence. So, it was decided to only cut the sequence short at the moment an 'N' character is registered.

BFAST algorithm

BASE type memory efficient

FTP for easy on and offloading the data

SEQ\_INDEX type maken memory efficient (75-300)