%%%%%%%%%%%%%%%%%%%%%%%%%%%%%%%%%%%%%%%%%%%%%%%%%%%%%%%%%%%%%%%%%%% 
%                                                                 %
%                            CHAPTER                              %
%                                                                 %
%%%%%%%%%%%%%%%%%%%%%%%%%%%%%%%%%%%%%%%%%%%%%%%%%%%%%%%%%%%%%%%%%%% 

\chapter{Structuur van de masterproeftekst}

\section{Opdeling in hoofdstukken}
De masterproeftekst vormt de kern van de scriptie. De tekst wordt logisch opgedeeld in een aantal hoofdstukken. Het eerste hoofdstuk is altijd een inleiding, het tweede en eventueel derde de literatuurstudie of een \textit{state of the art}, gevolgd door een hoofdstuk dat de methodologie beschrijft. De volgende hoofdstukken bevatten de elementen van het eigen onderzoek. Het laatste hoofdstuk bevat de algemene besluiten van de masterproef. Elk hoofdstuk vormt een afgerond geheel (m.a.w. met inleiding en conclusie!).

\section{Verdere onderverdeling binnen een hoofdstuk}
De tekst wordt onderverdeeld in logische paragrafen met een aangepaste nummering. De nummering van de onderliggende delen van een hoofdstuk bevat begint steeds met het hoofstuknummer en gaat maximum tot drie subniveaus. 
Volgende onderverdeling wordt gebruikt:

\section{Dit is een voorbeeld van een sectie}
\subsection{Dit is een voorbeeld van een subsectie}
\subsubsection{Dit is een voorbeeld van een subsubsectie}
\paragraph{Dit is een voorbeeld van een paragraaf}